% !TEX TS-program = pdflatex
% !TEX encoding = UTF-8 Unicode

% This is a simple template for a LaTeX document using the "article" class.
% See "book", "report", "letter" for other types of document.

\documentclass[11pt]{article} % use larger type; default would be 10pt

\usepackage[utf8]{inputenc} % set input encoding (not needed with XeLaTeX)

%%% Examples of Article customizations
% These packages are optional, depending whether you want the features they provide.
% See the LaTeX Companion or other references for full information.

%%% PAGE DIMENSIONS
\usepackage{geometry} % to change the page dimensions
\geometry{a4paper} % or letterpaper (US) or a5paper or....
 \geometry{margin=0.5in} % for example, change the margins to 2 inches all round
% \geometry{landscape} % set up the page for landscape
%   read geometry.pdf for detailed page layout information
\usepackage{graphicx} % support the \includegraphics command and options
% \usepackage[parfill]{parskip} % Activate to begin paragraphs with an empty line rather than an indent
%%% PACKAGES
\usepackage{booktabs} % for much better looking tables
\usepackage{array} % for better arrays (eg matrices) in maths
\usepackage{paralist} % very flexible & customisable lists (eg. enumerate/itemize, etc.)
\usepackage{verbatim} % adds environment for commenting out blocks of text & for better verbatim
\usepackage{subfig} % make it possible to include more than one captioned figure/table in a single float
\usepackage{times}
% These packages are all incorporated in the memoir class to one degree or another...
%\usepackage{biblatex} 
%\bibliography{Essay}

\usepackage{listings}
\usepackage{color}
\definecolor{dkgreen}{rgb}{0,0.6,0}
\definecolor{gray}{rgb}{0.5,0.5,0.5}
\definecolor{mauve}{rgb}{0.58,0,0.82}
\lstset{frame=tb,
  language=Python,
  aboveskip=3mm,
  belowskip=3mm,
  showstringspaces=false,
  columns=flexible,
  basicstyle={\small\ttfamily},
  numbers=none,
  numberstyle=\tiny\color{gray},
  keywordstyle=\color{blue},
  commentstyle=\color{dkgreen},
  stringstyle=\color{mauve},
  breaklines=true,
  breakatwhitespace=true
  tabsize=3
}
%%% HEADERS & FOOTERS
\usepackage{fancyhdr} % This should be set AFTER setting up the page geometry
\pagestyle{fancy} % options: empty , plain , fancy
\renewcommand{\headrulewidth}{0pt} % customise the layout...
\lhead{}\chead{}\rhead{}
\lfoot{}\cfoot{\thepage}\rfoot{}

%%% SECTION TITLE APPEARANCE
\usepackage{sectsty}
\allsectionsfont{\sffamily\mdseries\upshape} % (See the fntguide.pdf for font help)
% (This matches ConTeXt defaults)

%%% ToC (table of contents) APPEARANCE
\usepackage[titles,subfigure]{tocloft} % Alter the style of the Table of Contents
\renewcommand{\cftsecfont}{\rmfamily\mdseries\upshape}
\renewcommand{\cftsecpagefont}{\rmfamily\mdseries\upshape} % No bold!

%%% END Article customizations

%%% The "real" document content comes below...

\title{A Comparison and Analysis of Multithreading Language and Library Implementations and Features with Respect to Conway's Game of Life\\Final Year Project}
\author{Edward Michniak 10233252}
\date{} % Activate to display a given date or no date (if empty),
         % otherwise the current date is printed 

\begin{document}
\maketitle
\tableofcontents
\pagebreak
\section{Abstract}
\section{Introduction}
\section{Background Research and Required Knowledge}
\subsection{Life}
\begin{itemize}
\item The game at a glance
\subitem Define a start state
\subitem Begin automata etc...
\item What makes it so interesting?
\item Why code?
\subitem "The sheer number of different implementations of this game make it an extremely interesting case study, some languages may be better for one model than an another, an implementation may be possible in one language but not in another!(due to the available features)"



\end{itemize}
\subsection{Prototype and Experimental Research}
\begin{itemize}
\item Sequential C code?
\subitem Mention results
\item Defining parallel points in execution 
\subitem Reference parallel programming book
\item Building the abstract pseudo-algorithm for sequential execution
\item Requirements elicitation? (It has to go somewhere!)
\subitem Functional and non-functional requirements of language
\end{itemize}
\subsection{Threads}
\begin{itemize}
\item Von Neumann architecture
\subitem Von Neumann bottleneck?
\item Harvard Architecture
\subitem differences? Bottlnecks?
\item Sequential vs. Threaded
\item Common issues: Consumer-Producer, Cooperation, Race condition, dead lock, live lock.
\item What would happen if we didn't have the synchronisation directives in the code?
\item System kernel? How this effects the handling of threads.
\item Introduction to packages/languages that will be used.
\subitem Briefly explain the differences, look (very quickly) at some syntax.
\item sleep why is this bad?
\item thread states. ready. executing. dead. etc... HOW THE SYSTEM HANDLES THEM!
\subitem how does (generically) a language interface with a systems threads?
\item more needs to be here!
\end{itemize}
\section{Languages, Compilers, and Interpreters}
\begin{itemize}
\item Interesting point: TeX the package used to create documents is a form of compilation.
\item Introduce the languages more formally.
\subitem Is history necessary here? Java history is quite interesting. 
\subitem C99 not supported in windows. Worth mentioning?
\subitem Language type (Functional, procedural/imperative, OO)
\subitem Language features (typing, inheritance, classes, shared objects, evaluation, verbosity)
\item Explain the difference between a compiler and an interpreter (don't forget JIT). 
\subitem Maybe do this during the language introduction? 
\subitem Highlight which language uses which.
\subitem Mention optimisation, take a quick look at some of the C features that offer this (even the deprecated ones: register etc...)
\subitem how do interpreters handle optimisation?
\subitem how can they ever keep up with compiled languages?
\item Recommendations based on research
\subitem Is there one language that offers everything?
\subitem Could a desirable feature be made available to another language/library at little cost?
\end{itemize}
\section{Implementation}
\begin{itemize}
\item Challenges and appropriateness for implementation
\subitem How the programming paradigm (proc, oop, functional){\bf affects the life implementation?} Mainly mention limitations here! (In haskell we can do X, but in C we can do Y blah blah...) Mention typing, inheritance, evaluation, verbosity, classes, shared objects...etc...
\subitem Memory structures? (quad tree, sequential bits, OOP, struct linked list)
\subitem Processing methods (sequential, threaded, more threads, less threads, thread communication?)
\subitem FOR ALL OF THESE MENTION LIMITATIONS AND BENEFITS!
\item Pseudo-Code (both sequential and threaded)
\item UML (both sequential and threaded)
\subitem state diagram showing generation processes
\subitem class diagram for OOP models
\subitem SEQUENCE diagram!! Useful for showing method calls and generation order. Couple with state Diagram
\item Validation technique
\subitem Functional model (Possible haskell, lisp, clojure, ML, Maybe even SaC?!?)
\subitem Original seed grid
\item Program variables!
\subitem How will this effect runtime performance?
\item The Abstract pseudo Algorithm!!
\subitem ``Basic Explanation.txt''
\subsection{C with OpenMP}
\begin{itemize}
\item Language features
\subitem Register
\subitem Inline
\subitem Optimizer compiler
\subitem Precompiler macros
\item Library features
\subitem atomic
\subitem barrier
\subitem master (same as tid == 0)
\subitem single (different to tid == 0, one thread only!)
\subitem parallel
\subitem for (how could this change the code??)
\subitem ordered
\item POI!! Implementing different library directives changes the results drastically.... Why?! Look into trace thread? correction...Linux Trace Tool Kit
\item Does it conform to the abstract routine?
\item shared memory versus private memory implementation. Problems? Results? Terribleness! Include code. EVIDENCE!!
\end{itemize} 

OpenMP is an ``API specification for parallel programming'' ... \cite{ref3}

\subsection{C with PThreads}
\subsection{C++ with Boost}
\subsection{Java with Threads}
\begin{itemize}
\item gcj and javac
\item fork, join, yield, sleep
\item Overheads? Thread creation - compare to C
\item OOP? Threads as an object?
\end{itemize}
\subsection{Pyhton}
\begin{itemize}
\item List comprehension
\item Parallel processing?
\item parallel applications of Map, Reduce, and List Comprehension? Can it be done? Reference Parallel Computing: Architectures, Algorithms, and Applications, C. Bischof et al (Eds.) P203 Implementing Data-Parallel Patterns for Shared Memory with OpenMP GOOD ARTICLE!
\item The other paradigm? Fetching a 3x3 block of cells. Processing them, return the result... 
\end{itemize}
\subsection{C with CUDA}
\section{Analysis}
\subsection{Problems}
\begin{itemize}
\item Should this really be here? It should really be rounded up in each subheading...
\item Optimisations compiler results mess up
\item Functional model learning curve and paradigm shift. 
\item Python - Using functional techniques.
\end{itemize}
\subsection{Syntax and Semantics}
\subsection{Ease of use}
\subsection{Features}
\subsection{Performance}
\subsection{Extensions}
\begin{itemize}
\item Using OOP to recognize patterns and predict movement. I.e. Gliders and inverters.
\end{itemize}
\section{Conclusion}
Some sort of choice based on the most appropriate language.
\section{Bibliography and References}
\nocite{ref1}
\nocite{ref2}
\nocite{ref4}
\nocite{ref5}
\bibliographystyle{unsrt}
\bibliography{Project.bib}
\section{Appendices}
All the code!! With comments
\end{itemize}
\end{document}